% vim: ft=tex
%!TEX root=sbst2016.tex

\vspace*{-.5em}

\section{Shared Repositories}
\label{sec:repositories}

% vim: ft=tex
%!TEX root=sbst2016.tex

\begin{figure*}

  \begin{center}

    % SECTION: Define block styles

    \tikzstyle{repository} = [rectangle, draw, thick, double, fill=gray!10, text width=7em, text centered, rounded corners, minimum height=4em]
    \tikzstyle{repository-atdg} = [rectangle, draw, thick, double, fill=gray!30, text width=7em, text centered, rounded corners, minimum height=4em]
    \tikzstyle{repository-atsp} = [rectangle, draw, thick, double, fill=gray!60, text width=7em, text centered, rounded corners, minimum height=4em]
    \tikzstyle{report-atdg} = [rectangle, draw, thick, double, fill=gray!30, text width=7em, text centered, rounded corners, minimum height=4em]
    \tikzstyle{report-atsp} = [rectangle, draw, thick, double, fill=gray!60, text width=7em, text centered, rounded corners, minimum height=4em]
    \tikzstyle{report-atdg-result} = [rectangle, draw, thick, fill=gray!30, text width=7em, text centered, rounded corners, minimum height=4em]
    \tikzstyle{report-atsp-result} = [rectangle, draw, thick, fill=gray!60, text width=7em, text centered, rounded corners, minimum height=4em]
    \tikzstyle{process} = [rectangle, draw, thick, fill=gray!10, text width=7em, text centered, minimum height=4em]

    % SECTION: Define the line styles

    \tikzstyle{line} = [draw, thick, double, -stealth]
    \tikzstyle{extra-line} = [draw, thick, -stealth]

    \begin{tikzpicture}[node distance = 10em, auto]

      % SECTION: Create all of the nodes that will exist in the figure

      % REPOSITORIES
      \node [repository] (sbst-analysis) {sbst-analysis \\ \vspace*{.25em}\faCode~in \faGithubAlt};
      \node [repository-atdg, right of=sbst-analysis, yshift=4em] (atdg-analysis) {atdg-analysis \\ \vspace*{.25em} \faCode~+~\faDatabase~in \faGit};
      \node [repository-atsp, right of=sbst-analysis, yshift=-4em] (atsp-analysis) {atsp-analysis \\ \vspace*{.25em} \faCode~+~\faDatabase~in \faGit};

      % REPORTS
      \node [report-atdg, right of=atdg-analysis] (atdg-report) {atdg-report \\ \vspace*{.25em} \faCode~in \faGit};
      \node [report-atsp, right of=atsp-analysis] (atsp-report) {atsp-report \\ \vspace*{.25em} \faCode~in \faGit};

      % COMPILERS
      \node [process, right of=sbst-analysis, xshift=20em] (compiler) {RMarkdown \\ Compiler};

      % REPORT RESULTS
      \node [report-atdg-result, right of=compiler, yshift=4em] (atdg-report-result)
        {atsp-report-result \\ \vspace*{.25em} \faBarChart~+~\faColumns~+~\faCog};
      \node [report-atsp-result, right of=compiler, yshift=-4em] (atsp-report-result)
        {atsp-report-result \\ \vspace*{.25em} \faBarChart~+~\faColumns~+~\faCog};

      % SECTION: Create all of the paths between the nodes
      \path [line] (sbst-analysis) -- (atdg-analysis) node[draw=none,fill=none,font=\scriptsize,midway,below] {};
      \path [line] (sbst-analysis) -- (atsp-analysis) node[draw=none,fill=none,font=\scriptsize,midway,below] {};
      \path [line] (atdg-analysis) -- (atdg-report) node[draw=none,fill=none,font=\scriptsize,midway,below] {};
      \path [line] (atsp-analysis) -- (atsp-report) node[draw=none,fill=none,font=\scriptsize,midway,below] {};
      \path [extra-line] (atdg-report) -- (compiler);
      \path [extra-line] (atsp-report) -- (compiler);
      \path [extra-line] (compiler) -- (atdg-report-result);
      \path [extra-line] (compiler) -- (atsp-report-result);

      \end{tikzpicture}

    \end{center}

    \caption{Shared code repositories to support the analysis of data arising from experiments with search-based software testing
    tools.}~\label{fig:statistical_repositories}

  \end{figure*}


It is important to underscore the fact that this paper's goal is not to call into question the ways in which the
``Hitchhiker's Guide'' has benefited the SBST community. Rather, we intend to suggest that it is now time to build on
the noteworthy foundation set by Arcuri and Briand by encapsulating their theory and practical suggestions into free and
open-source software shared through GitHub.

Figure~\ref{fig:statistical_repositories} lays out our vision for the ``free vehicles'' that will improve the
methodological maturity of the ``hitchhikers'' in the SBST community. The core of our proposal is the ``sbst-analysis''
R package that is developed with the R ``devtools'' package and hosted on GitHub. R packages, as described and
implemented by Wickham, support the disciplined creation and delivery of self-contained R source
code~\cite{Wickham2015}. This repository will contain well-documented functions implementing all of the best practices
for the statistical analysis of the randomized algorithms used by SBST researchers.

Since the repository is publicly available, inquisitive researchers can easily clone it (as indicated by the
\codecopygit~annotation in Figure~\ref{fig:statistical_repositories}) and then study the functions to better understand
their operation and assumption(s). When new analysis ideas emerge (e.g., the transformations proposed by
Neumann~\etal~\cite{Neumann2015}), then the developers of these new methods can fork ``sbst-analysis'' (as shown by the
\codeforkgit~label), add new functions or modify the existing ones, and submit a merge request back to the maintainers
of sbst-analysis; once approved the new code will be available to the community.

