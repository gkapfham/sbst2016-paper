% vim: ft=tex
%!TEX root=sbst2016.tex

\section{Introduction}
\label{sec:introduction}

The field of search-based software testing (SBST) often involves the implementation and experimental evaluation of
algorithms that employ randomization. For instance, automated test data generation (ATDG) with the alternating variable
method, or \AVM, employs randomness when it restarts after not finding data that successfully meets the chosen
testing objectives~\cite{McMinn2015}. Or, a genetic algorithm performing automated test suite prioritisation (ATSP) that
reorders tests during regression testing will randomly mutate portions of a candidate test suite in order to
create the best ordering~\cite{Walcott2006}.

Scientists must carefully design and conduct the experiments evaluating these algorithms to ensure that they account for
any inherent randomness. It is additionally important that these scientists employ the right methods to analyse the results
from these experiments. In the year 2011, Arcuri and Briand published a conference paper outlining some practical
guidelines for using statistical methods to analyse randomized algorithms~\cite{Arcuri2011}, like those often used in
SBST. The journal version of this paper, entitled ``A Hitchhiker's Guide to Statistical Tests for
Assessing Randomized Algorithms in Software Engineering''~\cite{Arcuri2014}, expands on the earlier paper while still
providing practical ways to rigorously analysis empirical results.

It is hard to underestimate the ways in which these two papers have benefited the SBST community. For instance, many
SBST scientists now correctly use non-parametric \wilcoxon to perform hypothesis testing. To complement these
significance tests, many researchers in the field use the nonparametric \atwelve statistic of Vargha and Delaney
\cite{Vargha2000} to compute effect sizes, which determine the average probability that one approach ``out performs''
another. While these two papers have achieved laudable ends, we argue that the subtleties of various statistical
analyses might cause well-intentioned SBST researchers to make mistakes that compromise the validity of their empirical
results. To this end, we argue for the improvement of methodological maturity in the SBST field through the development
and use of share repositories of statistical code. That is, we note that the members of the SBST community ---
hitchhikers that we are --- need code ``vehicles'' to ensure that we attain to higher levels of maturity in our
statistical analyses.

