% vim: ft=tex
%!TEX root=sbst2016.tex

\vspace*{-1em}

\section{Hitchhikers' Dilemmas}
\label{sec:dilemma}

To demonstrate the need for that the SBST community has for shared repositories of statistical analysis code, we develop
two fictional examples loosely based on past experiences with the development and evaluation of algorithms that use
randomization.

First, an SBST researcher named Tom wants to calculate \atwelve~and remembers that there is some R code in the
``Hitchhiker's Guide'' paper. Yet, to his dismay, the paper does not include a self-contained function for computing an
effect size and furthermore, the provided code and equations are on separate pages of the paper. After puzzling over
curiosities such as the purpose of the {\tt seq\_along} function, Tom decides to see if there is an R package that
already provides a function for computing \atwelve~and discovers ``effsize''. Much to his chagrin, Tom realises that the
default code for \atwelve~is different from what Arcuri and Briand recommend. Wanting to ensure that he calculates the
effect size correctly, Tom writes tests to confirm that the code he thinks the ``Guide'' recommends is numerically
equivalent to that which is provided by the ``effsize'' package.

The completion of Tom's analysis is further delayed when his colleague suggests that he read a recent article by
Neumann~\etal~\cite{Neumann2015}, suggesting that Vargha-Delaney effect sizes be transformed to ensure that the correct
conclusions are reached. Since Tom is computing effect sizes for the execution timings of an ATDG algorithm, he realises
that, following Neumann~\etal's advice, he must transform his large data set. Since Tom is not an expert R programmer,
his attempt at a transformation is error-prone and his solution is slow. Although Tom has heard of the ``dplyr''
R package and the benefits it brings to data analysis, he faces a deadline and decides to submit his paper with
\atwelve~values based on un-transformed timings.

In the second scenario, an SBST researcher named Elaine wants to perform hypothesis testing for the data that she has
collected about a ATSP algorithm. In this data set, that was curated with assistance from her industrial partners, it is
possible to discern when one test suite ordering is better than another while the difference between values is not
meaningful. Elaine consults the ``Hitchhiker's Guide'' and correctly decides to use a non-parametric test, ultimately
picking the {\tt wilcox.test} function mentioned in Section 11 of the paper. After scanning the documentation for this
function, guessing that it is probably suitable for her purposes, and performing her statistical analysis, Elaine writes and
submits her paper. Owing, at least in part, to the fact that Section 11 of the ``Guide'' does not define the phrase
``interval-scale results'', Elaine later learns from the reviews of her paper that industry-collected data is ordinal and
as such she incorrectly applied {\tt wilcox.test}.


