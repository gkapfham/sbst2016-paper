% vim: ft=tex
%!TEX root=sbst2016.tex

\vspace*{-.5em}

\section{Practical Suggestions}
\label{sec:suggestions}

% each variable is a column, each observation is a row, and each type of observational unit is a table

Along with arguing that ``hitchikers need free vehicles'' --- or that SBST researchers need freely availble software as
a means for improving their statistical analyses --- this paper also puts forth several practical suggestions for
improving the functions ultimately furnished by repositories like sbst-analysis and atsp-analysis.

As already mentioned, Tom and Elaine should use the expressive and efficient functions in dplyr to summarise and
transform the data provided with their R packages. In support of their use of dplyr's functions, they should also ensure
that their data sets are organized in a ``tidy'' fashion where ``each variable is a column, each observation is a row,
and each type of observational unit is a table''~\cite{Wickham2014}. When Tom and Elaine perform a statistical analysis,
they should, whenever possible, transform it's output to a data structure amenable to further analysis by using R
packages such a ``broom''. Also, when SBST researchers add visualisation functions to their R packages, they should
consider the use of the sophisticated ``ggplot2'' package. Finally, SBST researchers should realise that defects in
their analysis functions are a threat to the validity of their empirical results and, as such, use R packages such as
``testthat'' to write test cases that establish a confidence in the correctness of their transformation, statistical
analysis, and graphing functions.


